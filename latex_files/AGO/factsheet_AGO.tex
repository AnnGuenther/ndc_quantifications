\documentclass[12pt]{article}

 %%%
 \usepackage[utf8]{inputenc}
 \usepackage{graphicx}
 \usepackage{xcolor}
 \usepackage{geometry}\geometry{a4paper, total={170mm,257mm}, left=20mm, top=20mm}

 \definecolor{PIKorange}{RGB}{227, 114, 34}
 \definecolor{PIKgray}{RGB}{142, 144, 143}

 \title{ \bfseries \color{PIKorange} Angola: information on national emissions, population and GDP, and mitigation targets}

 %%%

 \begin{document}

 \maketitle

 %%%
 \noindent \textbf{Authors:} \newline
 \indent Annika Guenther$^{1}$ \newline
 \indent Johannes Guetschow$^{1}$ \newline
 \noindent \textbf{Affiliations:} \newline
 \indent 1. Potsdam Institute for Climate Impact Research, Germany \newline
 \noindent \textbf{DOI:} [to be added] \newline

 \textbf{TODO}
 \begin{itemize}
 \item Table with info on target (main and reclass; emissions from NDC; target quantis + plot).
 \item GWP: NDC emissions coverted from AR2 to AR4 by national conversion factor (2010--2017, PRIMAP-hist v2.1).
 \item References!
 \end{itemize}

 %%%
 \section{Non-LULUCF emissions and socio-economic data}
 \label{sec:nonLULUCFSocioEco}
 With national emissions of 100.5~Mt CO$_2$eq, Angola contributed 0.2\% to global emissions in 2017, while in 2030 its share is estimated to decrease to 0.1\% (Table~\ref{tab:overview}).
 The estimates for 2030 are based on the downscaled SSP2\footnote{\textbf{SSPs}: Shared Socio-economic Pathways.
 Narratives and challenges to mitigation and adaptation: 
 SSP1: Sustainability, Taking the Green Road (low~/ low);
 SSP2: Middle of the Road (medium~/ medium);
 SSP3: Regional Rivalry, A Rocky Road (high~/ high);
 SSP4: Inequality, A Road Divided (low~/ high); and
 SSP5: Fossio-fuelled Development, Taking the Highway (high~/ low).} Middle of the Road marker scenario (dmSSP2), in which Angola is estimated to emit 105.9~Mt CO$_2$eq in 2030.
 That change in emissions would constitute an increase of 5.4\% compared to 2017. 
 The pathways dmSSP1--5 show a range of 101.6--147.8~Mt CO$_2$eq in 2030, and 82.9--219.4~Mt CO$_2$eq in 2050.
 The country's global rank in terms of total emissions per unit of GDP\footnote{\textbf{GDP}: Gross Domestic Product. 
 Throughout this document the GDP is given as GDP~PPP, with PPP being the Purchasing Power Parity.} was 55 in 2017, and 124 regarding the per-capita emissions (60 and 146 in 2030).
 In terms of accumulated historical emissions, Angola contributed to the global 1850--2017 emissions by 0.2\%. 
 When only accounting for the years 1990--2017, its contribution stays the same to 0.2\%.
 All of the emissions are presented following GWP~AR4\footnote{\textbf{Global Warming Potential (GWP)}: we use GWP values from the IPCC 4$^{th}$ Assessment Report (AR4). 
 They reflect the forcing potential of one kilogram of a gas' emissions in comparison to one kilogram of CO$_2$ (GWP$_{CO2}$ = 1). 
 The GWPs correspond to a 100-yr period and are for CH$_4$:~25, for N$_2$O:~298, for SF$_6$:~22800, and for NF$_3$:~17200. 
 For the basket of HFC-gases the GWPs from AR4 are in the range 4--14800, and for PFCs 7190--12200. 
 To assess emissions of several GHGs, their emissions are weighted by their respective GWPs and presented in CO$_2$ equivalents (CO$_2$eq).}, and exclude emissions from LULUCF\footnote{\textbf{LULUCF}: Land Use, Land-Use Change and Forestry. 
 Emissions from LULUCF are excluded throughout the document, unless stated otherwise.} (exclLU), and bunkers fuels\footnote{\textbf{Bunkers fuels}: emissions from international aviation and shipping.} emissions (exclBunkers).
 \begin{table}[htbp]
 \centering
 \caption{National emissions (dmSSP2), GDP and population for Angola, together with the emissions per unit of GDP and per capita emissions (all for 2017 and 2030). 
 Additionally, the global share and its rank are displayed.}
 \label{tab:overview}
 \begin{tabular}{l || l r l r r}
 \bfseries  & \bfseries Year & \bfseries Total & \bfseries Unit & \bfseries Glob. share & \bfseries Rank \tabularnewline \hline \hline
 \bfseries Emissions & 2017 & 100.5 & Mt CO$_2$eq & 0.2\% & 53 \tabularnewline 
 \bfseries  & 2030 & 105.9 & Mt CO$_2$eq & 0.1\% & 58 \tabularnewline \hline
 \bfseries GDP & 2017 & 146.0 & Billion 2011~GK\$ & 0.1\% & 73 \tabularnewline 
 \bfseries  & 2030 & 252.4 & Billion 2011~GK\$ & 0.1\% & 72 \tabularnewline \hline
 \bfseries Emissions & 2017 & 0.7 & t CO$_2$eq / Thousand 2011~GK\$ & 0.6\% & 55 \tabularnewline 
 \bfseries per GDP & 2030 & 0.4 & t CO$_2$eq / Thousand 2011~GK\$ & 0.6\% & 60 \tabularnewline \hline
 \bfseries Population & 2017 & 29.8 & Million Pers & 0.3\% & 45 \tabularnewline 
 \bfseries  & 2030 & 40.6 & Million Pers & 0.4\% & 40 \tabularnewline \hline
 \bfseries Emissions & 2017 & 3.4 & t CO$_2$eq /  Pers & 0.2\% & 124 \tabularnewline 
 \bfseries per capita & 2030 & 2.6 & t CO$_2$eq /  Pers & 0.1\% & 146 \tabularnewline 
 \end{tabular}
 \end{table}

 For Angola, in 2017 the main emissions share on sectoral level (Fig.~\ref{fig:tsEmi}) came from the Energy sector (56.7\%), followed by Agriculture (36.6\%), and Waste (5.1\%). 
 The Kyoto~GHG\footnote{\textbf{Kyoto~GHG} (Greenhouse Gas) basket: carbon dioxide (CO$_2$), methane (CH$_4$), nitrous oxide (N$_2$O), hydrofluorocarbons (HFCs), perfluorocarbons (PFCs), sulfur hexafluoride (SF$_6$), and nitrogen trifluoride (NF$_3$).} with the highest emissions in 2017 was CH$_4$, constituting  41.9\% of the national emissions. 
 Second largest contributor was CO$_2$ (38.6\%), followed by N$_2$O (19.4\%). 
 The total of F-gases\footnote{\textbf{F-gases} (fluorinated gases): basket of HFCs, PFCs, and the gases SF$_6$ and NF$_3$. 
 Some F-gases have very long atmospheric lifetimes and high Global Warming Potentials.} only represented 0.05\%.
 The total CO$_2$ emissions are expected to be 49.6\% of the national Kyoto~GHG emissions in 2030 (dmSSP2).
 \begin{figure}[htbp]
 \centering
 \includegraphics[width=\textwidth]{C:/Users/annikag/primap/ndc_quantifications/latex_files/AGO/ts_emi_exclLU_AGO.png}
 \caption{'Stacked' timeseries of national emissions (exclLU) per main-sector (a) and Kyoto~GHG (b). 
 No information available on the sectoral contributions after 2017.}
 \label{fig:tsEmi}
 \end{figure}
 \begin{figure}[htbp]
 \centering
 \includegraphics[width=\textwidth]{C:/Users/annikag/primap/ndc_quantifications/latex_files/AGO/ts_gdp_pop_AGO.png}
 \caption{Timeseries of national GDP (a) and population (c), and Kyoto~GHG emissions (exclLU, exclBunkers) per unit of GDP (b) or per capita (d).}
 \label{fig:tsSocioEco}
 \end{figure}

 The national GDP decreased in recent years, and the emissions per unit of GDP had a similar trend (Fig.~\ref{fig:tsSocioEco}).
 The population increased, while the per capita emissions were on the rise. 
 Following dmSSP2, the GDP is projected to increase towards 2050. 
 The emissions per GDP are estimated to decrease towars 2050. 
 Angola's population is assumed to grow towards 2050, and the per capita emissions are expected to increase after 2017 but to decline again before 2050. 

 %%%
 \section{LULUCF emissions}
 \label{sec:emiLULUCF}
 LULUCF emissions data for Angola are available from the following sources (Fig.~\ref{fig:tsLULUCF}): FAO (2019).

 \textbf{High fluctuations? Data gaps? Difference between sources?}
 \begin{figure}[htbp]
 \centering
 \includegraphics[width=\textwidth]{C:/Users/annikag/primap/ndc_quantifications/latex_files/AGO/ts_emi_onlyLU_AGO.png}
 \caption{Timeseries of emissions from LULUCF (CO$_2$ plus CH$_4$ and N$_2$O) as available from different data-sources. 
 Indicated in pink are the 'chosen' data, as used in our assessment of Angola's NDC (if needed). 
 The pink timeseries was inter- and~/ or extrapolated (interpolation: linear, extrapolation: constant).}
 \label{fig:tsLULUCF}
 \end{figure}

 %%%
 \section{Mitigation targets (NDC)}
 \label{sec:mitiTars}

 \textbf{ 
 Give the \%cov for the base and target year (and 2017).
 Global share for 2030 for the mitigated pathways and \% reduction relative to 1990 and 2017.
 Table with the 'input' data and the resulting targets (like ndcs\_targets.csv).}
 Angola has an INDC, with a GHG mitigation target of the type RBU (relative reduction compared to Business-As-Usual; main target type).
 The reclassified\footnote{\textbf{Reclassification}: when a country has, e.g., an RBU target (relative reduction compared to Business-As-Usual), and the BAU emissions are provided, it can be quantified based on the given emissions, and is reclassified from type\_main~RBU to type\_reclass~ABS (absolute emissions target).
 Additionally, 'NGT' targets can be reclassified as 'ABU' (absolute reduction compared to Business-As-Usual) if absolute mitigation effects due to planned policies and measures are provided.}  target type is ABS (absolute emissions target).
 \begin{table}[htbp]
 \centering
 \caption{Information on Angola's GHG mitigation target(s).}
 \label{tab:mitiTars}
 \begin{tabular}{l l l l l l }
 \bfseries type & \bfseries condi. & \bfseries range & \bfseries value & \bfseries tarYr & \bfseries LU \tabularnewline \hline
 RBU & uncondi. & best & -20\% & 2025 & inclLU \tabularnewline 
 RBU & uncondi. & best & -35\% & 2030 & inclLU \tabularnewline 
 RBU & condi. & best & -27\% & 2025 & inclLU \tabularnewline 
 RBU & condi. & best & -50\% & 2030 & inclLU \tabularnewline 
 ABS & uncondi. & best & 124.65 Mt~CO$_2$eq AR4 & 2025 & inclLU \tabularnewline 
 ABS & uncondi. & best & 125.61 Mt~CO$_2$eq AR4 & 2030 & inclLU \tabularnewline 
 ABS & condi. & best & 113.74 Mt~CO$_2$eq AR4 & 2025 & inclLU \tabularnewline 
 ABS & condi. & best & 96.62 Mt~CO$_2$eq AR4 & 2030 & inclLU \tabularnewline 
 \end{tabular}
 \end{table}
 \begin{figure}[htbp]
 \centering
 \includegraphics[width=\textwidth]{C:/Users/annikag/primap/ndc_quantifications/latex_files/AGO/ts_pc_cov_AGO.png}
 \caption{Timeseries of Angola's national emissions (exclLU) and the share of emissions that is assumed to be covered by Angola's mitigation target.}
 \label{fig:tsPcCov}
 \end{figure}
 \begin{table}[htbp]\small
 \centering
 \caption{Information on covered sectors and gases as retrieved from INDC and adapted ('Adap.': used to calculate \%cov), and their shares in Angola's 2017 emissions (exclLU, exclBunkers; total 100.5~Mt CO$_2$eq).
 If either the sector or gas is assessed as 'not-covered', the emissions from this sector-gas combination are counted as not-covered (--). 
 Else the emissions are counted as covered (+; covered shares given in bold).
 (/) means that no information is available.
 LULUCF: INDC '+' and adapted '+' (estimated as a net source of 106.1~Mt CO$_2$eq in 2017; based on the 'chosen' LULUCF emissions).}
 \label{tab:coveredSectorsGases}
 \begin{tabular}{l || c c || c c c c c c c | c}
 \bfseries  & \bfseries NDCs & \bfseries Adap. & \bfseries CO$_2$ & \bfseries CH$_4$ & \bfseries N$_2$O & \bfseries HFCs & \bfseries PFCs & \bfseries SF$_6$ & \bfseries NF$_3$ & \bfseries Total \tabularnewline \hline \hline
 \bfseries NDCs &  &  & \bfseries + & \bfseries + & \bfseries + & -- & -- & -- & -- &  \tabularnewline 
 \bfseries Adap. &  &  & \bfseries + & \bfseries + & \bfseries + & -- & -- & -- & -- &  \tabularnewline \hline \hline
 \bfseries Energy & \bfseries + & \bfseries + & \bfseries 37.6\% & \bfseries 18.7\% & \bfseries 0.4\% & / & / & / & / & 56.7\% \tabularnewline 
 \bfseries IPPU & \bfseries + & \bfseries + & \bfseries 1.0\% & \bfseries / & \bfseries 0.1\% & 0.05\% & / & / & / & 1.2\% \tabularnewline 
 \bfseries Agri. & \bfseries + & \bfseries + & \bfseries / & \bfseries 18.4\% & \bfseries 18.2\% & / & / & / & / & 36.6\% \tabularnewline 
 \bfseries Waste & \bfseries + & \bfseries + & \bfseries / & \bfseries 4.8\% & \bfseries 0.2\% & / & / & / & / & 5.1\% \tabularnewline 
 \bfseries Other & / & \bfseries + & \bfseries / & \bfseries / & \bfseries 0.3\% & / & / & / & / & 0.3\% \tabularnewline \hline
 \bfseries Total &  &  & 38.6\% & 41.9\% & 19.4\% & 0.05\% & / & / & / & 100.0\% \tabularnewline 
 \end{tabular}
 \end{table}
 \begin{figure}[htbp]
 \centering
 \includegraphics[width=\textwidth]{C:/Users/annikag/primap/ndc_quantifications/latex_files/AGO/ts_ndcs_quantis_AGO.png}
 \caption{Quantified mitigation targets (based on different input data and calculation options).
 Vertical lines: conditionality~/ range;
 colour coded: dmSSP1--5;
 first~/ second set of six: prio NDCs~/ SSPs;
 set of six: coverage 100, lulucf unfccc, lulucf fao, bl uncondi, const emi, estimated coverage.}
 \label{fig:miti}
 \end{figure}

 %%%
 \section{Data sources and references}
 \label{sec:dataSourcesRefs}
 PRIMAP-hist v2.1: emissions from PRIMAP-hist are data from the country reported data priority scenario (HISTCR).

 %%%
 \end{document}
