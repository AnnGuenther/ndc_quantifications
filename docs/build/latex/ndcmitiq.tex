%% Generated by Sphinx.
\def\sphinxdocclass{report}
\documentclass[letterpaper,10pt,english]{sphinxmanual}
\ifdefined\pdfpxdimen
   \let\sphinxpxdimen\pdfpxdimen\else\newdimen\sphinxpxdimen
\fi \sphinxpxdimen=.75bp\relax

\PassOptionsToPackage{warn}{textcomp}
\usepackage[utf8]{inputenc}
\ifdefined\DeclareUnicodeCharacter
% support both utf8 and utf8x syntaxes
  \ifdefined\DeclareUnicodeCharacterAsOptional
    \def\sphinxDUC#1{\DeclareUnicodeCharacter{"#1}}
  \else
    \let\sphinxDUC\DeclareUnicodeCharacter
  \fi
  \sphinxDUC{00A0}{\nobreakspace}
  \sphinxDUC{2500}{\sphinxunichar{2500}}
  \sphinxDUC{2502}{\sphinxunichar{2502}}
  \sphinxDUC{2514}{\sphinxunichar{2514}}
  \sphinxDUC{251C}{\sphinxunichar{251C}}
  \sphinxDUC{2572}{\textbackslash}
\fi
\usepackage{cmap}
\usepackage[T1]{fontenc}
\usepackage{amsmath,amssymb,amstext}
\usepackage{babel}



\usepackage{times}
\expandafter\ifx\csname T@LGR\endcsname\relax
\else
% LGR was declared as font encoding
  \substitutefont{LGR}{\rmdefault}{cmr}
  \substitutefont{LGR}{\sfdefault}{cmss}
  \substitutefont{LGR}{\ttdefault}{cmtt}
\fi
\expandafter\ifx\csname T@X2\endcsname\relax
  \expandafter\ifx\csname T@T2A\endcsname\relax
  \else
  % T2A was declared as font encoding
    \substitutefont{T2A}{\rmdefault}{cmr}
    \substitutefont{T2A}{\sfdefault}{cmss}
    \substitutefont{T2A}{\ttdefault}{cmtt}
  \fi
\else
% X2 was declared as font encoding
  \substitutefont{X2}{\rmdefault}{cmr}
  \substitutefont{X2}{\sfdefault}{cmss}
  \substitutefont{X2}{\ttdefault}{cmtt}
\fi


\usepackage[Bjarne]{fncychap}
\usepackage{sphinx}

\fvset{fontsize=\small}
\usepackage{geometry}


% Include hyperref last.
\usepackage{hyperref}
% Fix anchor placement for figures with captions.
\usepackage{hypcap}% it must be loaded after hyperref.
% Set up styles of URL: it should be placed after hyperref.
\urlstyle{same}
\addto\captionsenglish{\renewcommand{\contentsname}{Contents:}}

\usepackage{sphinxmessages}
\setcounter{tocdepth}{1}



\title{NDCmitiQ}
\date{Jul 22, 2020}
\release{1.0}
\author{Annika Guenther}
\newcommand{\sphinxlogo}{\vbox{}}
\renewcommand{\releasename}{Release}
\makeindex
\begin{document}

\pagestyle{empty}
\sphinxmaketitle
\pagestyle{plain}
\sphinxtableofcontents
\pagestyle{normal}
\phantomsection\label{\detokenize{index::doc}}


\sphinxstyleemphasis{Comments: in the code, use (3 times ” newline text newline 3 times “) for comments.} \sphinxstyleemphasis{Do not forget the new lines, else it will not appear in this documentation.}

This documentation of the main functions to preprocess data and quantify mitigation
targets from within the NDCs (tool NDCmitiQ), includes information retrieved from the
different py\sphinxhyphen{}files of the tool.
It does not include information from all py\sphinxhyphen{}files in this repository.

This documentation shall be seen as an add\sphinxhyphen{}on to the information given in the
manuscript describing the methodology behind NDCmitiQ.
As soon as the paper is published, its DOI will be added here.

The required pandas packages can be found in requirements.txt,
and information on how to run the code can be found in the README in the main folder.


\chapter{setup\_metadata}
\label{\detokenize{code:setup-metadata}}\label{\detokenize{code::doc}}
\sphinxstylestrong{setup\_metadata()}
\begin{quote}

Set up the general metadata.
If you want/need to change the folder for the preprocessed data, do it here.

Returns:
\begin{description}
\item[{meta}] \leavevmode{[}class.{]}
Meta data needed in the quantifications and plotting routines.

\end{description}

meta.path: paths to different folders.

meta.isos: iso3 codes of EARTH, EU, conversion from iso3 to short country names.

SSPs: Shared Socioeconomic Pathways; info on available marker scenarios.

Nomenclature: table meta\sphinxhyphen{}data nomenclature.

PRIMAP\sphinxhyphen{}hist: info on sources and scenarios (emi, pop, gdp).

Default\sphinxhyphen{}units

GWPs

LULUCF source\sphinxhyphen{}priorisation

Gases: basket\sphinxhyphen{}members, labels.

Sectors: main sectors, labels.

Categories: main categories, labels.

Nice labels for sources

NDC types.
\end{quote}


\chapter{preprocessing\_general}
\label{\detokenize{code:preprocessing-general}}
General preprocessing (covered part of emissions in preprocessing\_current\_pccov.py).

Fill missing data, SSPs (emi, pop, gdp), LULUCF emissions.

Load and write out information retrieved from NDCs to ‘ndcs\_info.csv’ into folder meta.path.preprocess.
Includes information on Parties’ target types, years, conditionality, range, coverage.

Read in all datatables in folder matlab\_tables.

All tables are read into the class ‘database’, with the tablenames as main\sphinxhyphen{}attributes.
The table meta\sphinxhyphen{}data are classes again, with attributes ‘entity’, ‘category’,
‘data’, ‘unit’, etc. (see meta.nomenclature.attrs).

As we only use one GWP, replace the AR4 strings by ‘’
(e.g., KYOTOGHGAR4\_IPCM0EL\_TOTAL\_NET\_HISTORY\_CRF2019.csv is read in
with tablename ‘KYOTOGHG\_IPCM0EL\_TOTAL\_NET\_HISTORY\_CRF2019’).

SSPs: check if there are countries that do have data in some SSPs but not in others and use the average over the available SSPs for the non\sphinxhyphen{}available SSPs.
For countries that have PRIMAP\sphinxhyphen{}hist data but no SSP data at all (for marker scenarios of SSP1 to SSP5), use the PRIMAP\sphinxhyphen{}hist data and use the linear regression over the last 6 available years as future estimates.
Only happens for countries with very low emissions.

The SSPs only have data for FGASES, not separated into HFCS, PFCS, SF6 and NF3.
For the calculation of the future pc\_cov, a split into the four subgroups is performed.
The historical share per gas/basket is kept constant and applied to the future FGASES\sphinxhyphen{}basket.
The mean over the last 6 available years is used.

Check if the SSPs (KYOTOGHG\_IPCM0EL per SSP scenario) are consistent with the given PRIMAP\sphinxhyphen{}hist data.

Prepare LULUCF data.
Problems with LULUCF data: high inter\sphinxhyphen{}annual variability, negative / positive emissions,
and the data we use are not consistent with the time series used in the pathway extension.

Make one LULUCF table, with the chosen KYOTOGHG\_IPCMLULUCF time series.
Prioritisation of data\sphinxhyphen{}sources as given in meta.lulucf.source\_prioritisation.
When a source has data, use them, and fill data gaps with ‘constant filling’.

Interpolation \& forward extrapolation: last value kept constant.
Backward extrapolation: first available value after 1990 used for 1990 to year of first available value).

Other option: linear interpolation, but constant extrapolation (with mean over several years).

If KYOTOGHG is calculated here as sum over CO2 + CH4 + N2O: sum up the already inter\sphinxhyphen{} \& extrapolated time series.

Additionally to prioritising CRF, BRU, UNFCCC, FAO in that order, put FAO or UNFCCC to the first place in the LULUCF prioritisation, for comparison runs.

Create NDC exclLU pathways.
Use PRIMAP\sphinxhyphen{}hist v2.1 HISTCR up to 2017 and then use given ndcs\_emi\_exclLU when available, with linear interpolation between values.

For the countries without exclLU values check if they have inclLU values.
If so use the onlyLU data from the NDC (or other options, as used in the NDC quantifications) to derive onlyLU emissions.
Use the given inclLU values and the corresponding onlyLU values to derive exclLU, and then interpolate linearly between them (again, with PRIMAP\sphinxhyphen{}hist for KYOTOGHG\_IPCM0EL up to 2017).
\begin{quote}

EMI onlyLU.
\begin{quote}

If there are onlyLU values available for 2010 or later, use them for constant extrapolation (use their average) if no future values are available.

EMI exclLU.

Check if there is exclLU data available from the NDC for years \textgreater{} 2017.
If so, use them with linear interpolation from PRIMAP\sphinxhyphen{}hist to the first year, and inbetween.
After last year: use SSP growth rates.
\begin{itemize}
\item {} 
If ndc\_exclLU available: use ndc\_exclLU.

\item {} 
If ndc\_inclLU available: use ndc\_inclLU \sphinxhyphen{} onlyLU.

\item {} 
Else: use external\_exclLU.

\end{itemize}

Put in PRIMAP\sphinxhyphen{}hist data to 1990 \sphinxhyphen{} 2017.
And interpolate linearly.
Extrapolate with the SSP growth rates.
\end{quote}
\end{quote}


\chapter{preprocessing.prep\_read\_in\_tables}
\label{\detokenize{code:preprocessing-prep-read-in-tables}}
\sphinxstylestrong{prep\_read\_in\_tables(file, path\_to\_folder, database, meta)}
\begin{quote}

Read in all datatables in folder.

All read into class ‘database’, with the tablenames as attributes.
Replace AR4 strings by ‘’
(e.g., KYOTOGHGAR4\_IPCM0EL\_TOTAL\_NET\_HISTORY\_CRF2019.csv is read in
with tablename ‘KYOTOGHG\_IPCM0EL\_TOTAL\_NET\_HISTORY\_CRF2019’).
The attributes are classes again, with attributes entity, category,
data, family, etc. (see meta.nomenclature.attrs).
\end{quote}


\chapter{preprocessing.prep\_ssps\_fill\_gaps}
\label{\detokenize{code:preprocessing-prep-ssps-fill-gaps}}
\sphinxstylestrong{prep\_ssps\_fill\_gaps(database, info\_per\_country, meta, nrvalues)}
\begin{quote}

SSPs: check if there are countries that do have data in some SSPs but not in others and
use the average over the available SSPs for the non\sphinxhyphen{}available SSPs.
For countries that have PRIMAP\sphinxhyphen{}hist data but no SSP data at all (for SSP1 to SSP5),
use the PRIMAP\sphinxhyphen{}hist data and use the linear regression over the last 6 available years
as future estimates.
Only happens for countries with very low emissions.
\end{quote}

\sphinxstylestrong{fill\_values(database, info\_per\_country, info\_act,         primap\_extrapol, nrvalues, ent, ssp, ssps\_test, ssps\_mean)}
\begin{quote}

For countries for which there are data available for some SSPs, but not for others,
use the average over the available SSPs for the non\sphinxhyphen{}available SSPs.
And for countries that have PRIMAP\sphinxhyphen{}hist data, but no SSP data at all, use the PRIMAP\sphinxhyphen{}hist
data and use the linear regression over the last 6 available years as future estimates.
\end{quote}


\chapter{preprocessing.prep\_ssps\_split\_fgases}
\label{\detokenize{code:preprocessing-prep-ssps-split-fgases}}
\sphinxstylestrong{prep\_ssps\_split\_fgases(meta, database, nrvalues)}
\begin{quote}

The SSPs only have data for FGASES, not separated into HFCS, PFCS, SF6 and NF3.
For the calculation of the future pc\_cov, a split into the four subgroups is performed.
The historical share per gas/basket is kept constant and applied to the future FGASES\sphinxhyphen{}basket.
The mean over the last 6 available years is used.
\end{quote}

\sphinxstylestrong{calc\_and\_store\_data(database, meta, fgases\_share, nrvalues)}


\chapter{preprocessing.prep\_lulucf}
\label{\detokenize{code:preprocessing-prep-lulucf}}
\sphinxstylestrong{prep\_lulucf(database, meta, prios, srce\_name, info\_per\_country, nrvalues, interpolation\_method)}
\begin{quote}

Prepare LULUCF data.

Make one LULUCF table, with the chosen KYOTOGHG\_IPCMLULUCF time series.
Prioritisation of data\sphinxhyphen{}sources as given in prios.
When a source has data, use them, and fill data gaps with ‘constant filling’
(interpolation \& forward extrapolation: mean over last values kept constant,
backward extrapolation: mean over first available values kept constant).
If KYOTOGHG is calculated here from CO2 + CH4 + N2O: sum up the already inter\sphinxhyphen{} \& extrapolated time series.

Problems with LULUCF data: high inter\sphinxhyphen{}annual variability, negative / positive emissions,
and the data we use are not consistent with the time series used in the pathway extension.

interpolation\_method: ‘constant’ or ‘linear’.
\end{quote}

\sphinxstylestrong{calc\_data()}
\begin{quote}

For the future, use the mean over 2010 to whatever the most recent historical value is,
or if no values are available starting from 2010, use the last value.
For the backward extrapolation use the mean over 1990 to 1997, or the first available value.
\end{quote}

\sphinxstylestrong{store\_data()}
\begin{quote}
\begin{quote}

Only use a source if at least 6 values are available for 1990 \sphinxhyphen{} 2017.
If no other source has data, then use a source with less than 6 values available nevertheless.
Store the DataFrames with various sources combined to one datatable in lulucf\_table.
\end{quote}

Only use a source if at least 6 values are available for 1990 \sphinxhyphen{} 2017.
If no other source has data, then use a source with less than 6 values available nevertheless.
Store the DataFrames with various sources combined to one datatable in lulucf\_table.
\end{quote}


\chapter{preprocessing\_current\_pc\_cov}
\label{\detokenize{code:preprocessing-current-pc-cov}}
Preprocessing to calculate the part of emissions covered by national mitigation targets.

Read in all datatables in folder meta.path.preprocess/tables.

All read into class ‘database’, with the tablenames as attributes.
The meta\sphinxhyphen{}data are classes again, with attributes entity, category, data, unit, etc. (see meta.nomenclature.attrs).

Calculate the part of historical emissions that is covered by an NDC.

If the country does not have an (I)NDC: nothing is covered.

Else:
\begin{itemize}
\item {} 
Assessment based on PRIMAP\sphinxhyphen{}hist HISTCR emissions time series.

\item {} 
Categories and gases assessed (per country):
\begin{itemize}
\item {} 
Main categories (IPC1, IPC2, IPCMAG, IPC4 and IPC5; namely Energy, IPPU, Agriculture, Waste and Other; excludes LULUCF).

\item {} 
Kyoto GHGs: CO2, CH4, N2O, HFCS, PFCS, SF6, NF3.

\end{itemize}

\item {} 
For each of these categories / gases, the information on whether they are covered by the country’s NDC is provided (csv\sphinxhyphen{}input, assessed by A. Günther).

\item {} 
If no information is available for all gases: CO2, CH4, and N2O are assumed to be covered (in the csv\sphinxhyphen{}file already).

\item {} 
If any of HFCS, PFCS, SF6 or NF3 is covered, put IPPU to covered (F\sphinxhyphen{}gases only relevant in IPC2).

\item {} 
If all sectors (IPC1, 2, MAG, 4) are covered, the category “Other” (IPC5) is set to “YES” (in the csv\sphinxhyphen{}file already).

\item {} 
For all category + gas combinations, the emissions are counted as covered, if neither the category nor the gas are assumed not to be covered (neither category nor gas can contain a “NO”).

\end{itemize}

Here, matrices on the coverage (Yes: covered, No: not\sphinxhyphen{}covered) are created.

We do not include the information on the EU28, but put the information into each of the member\sphinxhyphen{}states.

You can chose to setup your own coverage.used\_per\_gas\_per\_sec.
If you do so, write out a file to say what you have chosen!!!

E.g. put all Energy and CO2 to covered (‘YES’), and the rest to ‘NO’.
Or: all ANNEX\sphinxhyphen{}I parties cover everything, and the rest only Energy and CO2.

The covered part of emissions (in ‘emissions’) is calculated here for historical years, for which data per sector and gas combination are available.
The assessment is based on the information in coverage.used\_per\_combi (all entries are already ‘YES’ or ‘NO’, no ‘NAN’ entries).
All combinations with ‘YES’ are summed up to emicov\_his, and all combinations with ‘NO’ are summed up to eminotcov\_his.

For KYOTOGHG\_IPCM0EL, excluding LULUCF.

Additionally calculate the historical emissions not\sphinxhyphen{}/covered for perGas\_IPCM0EL and KYOTOGHG\_perCategory.
These values are not needed in further calculations, but nice to have.

Calculate the part of future emissions covered by an NDC (KYOTOGHG\_IPCM0EL)
\begin{itemize}
\item {} 
For countries that cover everything: set pccov\_fut to 1 (100\%).

\item {} 
For countries that cover nothing: set pccov\_fut to 0 (0\%).

\item {} \begin{description}
\item[{For countries that cover all sectors (excl. LULUCF), but not all gases:}] \leavevmode
the SSP entity\_IPCM0EL emissions per gas are used to calculate pc\_cov\_fut.
\begin{itemize}
\item {} 
SSP data are available for KYOTOGHG, CO2, CH4, N2O and FGASES:
\begin{itemize}
\item {} 
For countries that cover only some FGASES, the \% share between HFCS, PFCS, SF6 and NF3 is kept constant (at mean over last 6 available PRIMAP\sphinxhyphen{}hist values).

\item {} 
The share per gas is applied to the future KYOTOGHG\_IPCM0EL emissions data.

\end{itemize}

\end{itemize}

\end{description}

\item {} 
For countries that do not cover all sectors:
\begin{itemize}
\item {} 
Calculate the slope of pc\_cov\_his (2010 to most recent year with available data (“mry”)).
\begin{itemize}
\item {} 
If abs(slope) \textless{} lim\_slope: use the mean over 2010 to mry.

\item {} 
If abs(slope) \textgreater{} lim\_slope: calculate pc\_cov\_fut from the correlation between emi\_tot\_his and emi\_cov\_his. For 2010 to mry.
\begin{itemize}
\item {} 
If any(pc\_cov\_fut) \textgreater{} 90\%, but not all(pc\_cov\_fut) \textgreater{} 90\% \textendash{}\textgreater{} set the pc\_cov\_fut \textgreater{} 90\% to 90\%.

\item {} 
If any(pc\_cov\_fut) \textless{} 10\%, but not all(pc\_cov\_fut) \textless{} 10\% \textendash{}\textgreater{} set the pc\_cov\_fut \textless{} 10\% to 10\%.

\item {} 
If any(pc\_cov\_fut) \textgreater{} 100\% or \textless{} 0\% use the mean instead.

\end{itemize}

\end{itemize}

\end{itemize}

\item {} 
If no future emissions data are available: use the mean over 2010 to mry.

\end{itemize}

The future emicov / pccov values depend on the chosen SSP scenario.

One can give a preference for the calculation method of pccov\_fut.
preference\_pccov\_fut can be ‘mean’ or ‘corr’.
\begin{description}
\item[{‘mean’:}] \leavevmode
Check for the countries for which the slope of a regression to the last available years of pccov\_his is less than slope\_lim.
Use the mean over the years as pccov\_fut.
For the others check the correlation between emitot and emicov and decide whether to better use this correlation for pccov\_fut.

\item[{‘corr’:}] \leavevmode
Take the correlation between emitot and emicov, unless it is too ‘bad’, then take the mean.

\end{description}

Default: ‘corr’.

Update the current pc\_cov\sphinxhyphen{}folder in setup\_metadata after running preprocessing\_current\_pc\_cov.py.


\chapter{preprocessing.prep\_coverage}
\label{\detokenize{code:preprocessing-prep-coverage}}
\sphinxstylestrong{prep\_coverage(meta, infos\_from\_ndcs, info\_per\_country)}
\begin{quote}

Calculate the part of historical emissions that is covered by an NDC.

If the country does not have an (I)NDC: nothing is covered.

Else:
\begin{itemize}
\item {} 
Assessment based on PRIMAP\sphinxhyphen{}hist HISTCR emissions time series.

\item {} 
Categories and gases assessed (per country):
\begin{itemize}
\item {} 
Main categories (IPC1, IPC2, IPCMAG, IPC4 and IPC5; namely Energy, IPPU, Agriculture, Waste and Other; excludes LULUCF).

\item {} 
Kyoto GHGs: CO2, CH4, N2O, HFCS, PFCS, SF6, NF3.

\end{itemize}

\item {} 
For each of these categories / gases, the information on whether they are covered by the country’s NDC is provided (csv\sphinxhyphen{}input, assessed by A. Günther).

\item {} 
If no information is available for all gases: CO2, CH4, and N2O are assumed to be covered (in the csv\sphinxhyphen{}file already).

\item {} 
If any of HFCS, PFCS, SF6 or NF3 is covered, put IPPU to covered (F\sphinxhyphen{}gases only relevant in IPC2).

\item {} 
If all sectors (IPC1, 2, MAG, 4) are covered, the category “Other” (IPC5) is set to “YES” (in the csv\sphinxhyphen{}file already).

\item {} 
For all category + gas combinations, the emissions are counted as covered, if neither the category nor the gas are assumed not to be covered (neither category nor gas can contain a “NO”).

\end{itemize}

Here, matrices on the coverage (Yes: covered, No: not\sphinxhyphen{}covered) are created.

We do not include the information on the EU28, but put the information into each of the member\sphinxhyphen{}states.
\end{quote}

\sphinxstylestrong{current\_coverage(meta, iso3, coverage, info\_per\_country)}


\chapter{preprocessing.prep\_covered\_emissions\_his}
\label{\detokenize{code:preprocessing-prep-covered-emissions-his}}
\sphinxstylestrong{prep\_covered\_emissions\_his(database, coverage, meta, primap)}
\begin{quote}

The covered part of emissions (in ‘emissions’) is calculated here for historical years,
for which data per sector and gas combination are available.
The assessment is based on the information in coverage.used\_per\_combi (all entries are ‘YES’ or ‘NO’).
All combinations with ‘YES’ are summed up to emicov\_his, and all combinations with ‘NO’ are summed up to eminotcov\_his.

For KYOTOGHG\_IPCM0EL, excluding LULUCF.
\end{quote}

\sphinxstylestrong{testing(new\_kyoto\_ipcm0el, database, meta)}
\begin{quote}

Test if the sum over the covered and not\sphinxhyphen{}covered emissions (KYOTOGHG\_IPCM0EL)
sum up to the original KYOTOGHG\_IPCM0EL in ‘database’.
\end{quote}


\chapter{preprocessing.prep\_covered\_emissions\_fut}
\label{\detokenize{code:preprocessing-prep-covered-emissions-fut}}
\sphinxstylestrong{prep\_covered\_emissions\_fut(database, meta, coverage, info\_per\_country, preference\_pccov\_fut, primap,     first\_year\_for\_slope, slope\_lim, rvalue\_lim)}
\begin{quote}

Calculate the part of future emissions covered by an NDC (KYOTOGHG\_IPCM0EL)
\begin{itemize}
\item {} 
For countries that cover everything: set pccov\_fut to 1 (100\%).

\item {} 
For countries that cover nothing: set pccov\_fut to 0 (0\%).

\item {} \begin{description}
\item[{For countries that cover all sectors (excl. LULUCF), but not all gases:}] \leavevmode
the SSP entity\_IPCM0EL emissions per gas are used to calculate pc\_cov\_fut.
\begin{itemize}
\item {} 
SSP data are available for KYOTOGHG, CO2, CH4, N2O and FGASES:
\begin{itemize}
\item {} 
For countries that cover only some FGASES, the \% share between HFCS, PFCS, SF6 and NF3 is kept constant (at mean over last 6 available PRIMAP\sphinxhyphen{}hist values).

\item {} 
The share per gas is applied to the future KYOTOGHG\_IPCM0EL emissions data.

\end{itemize}

\end{itemize}

\end{description}

\item {} 
For countries that do not cover all sectors:
\begin{itemize}
\item {} 
Calculate the slope of pc\_cov\_his (2010 to most recent year with available data (“mry”)).
\begin{itemize}
\item {} 
If abs(slope) \textless{} lim\_slope: use the mean over 2010 to mry.

\item {} 
If abs(slope) \textgreater{} lim\_slope: calculate pc\_cov\_fut from the correlation between emi\_tot\_his and emi\_cov\_his. For 2010 to mry.
\begin{itemize}
\item {} 
If any(pc\_cov\_fut) \textgreater{} 90\%, but not all(pc\_cov\_fut) \textgreater{} 90\% \textendash{}\textgreater{} set the pc\_cov\_fut \textgreater{} 90\% to 90\%.

\item {} 
If any(pc\_cov\_fut) \textless{} 10\%, but not all(pc\_cov\_fut) \textless{} 10\% \textendash{}\textgreater{} set the pc\_cov\_fut \textless{} 10\% to 10\%.

\item {} 
If any(pc\_cov\_fut) \textgreater{} 100\% or \textless{} 0\% use the mean instead.

\end{itemize}

\end{itemize}

\end{itemize}

\item {} 
If no future emissions data are available: use the mean over 2010 to mry.

\end{itemize}

The future emicov / pccov values depend on the chosen SSP scenario.

One can give a preference for the calculation method of pccov\_fut.
preference\_pccov\_fut can be ‘mean’ or ‘corr’.
\begin{description}
\item[{‘mean’:}] \leavevmode
Check for the countries for which the slope of a regression to the last available years of pccov\_his is less than slope\_lim.
Use the mean over the years as pccov\_fut.
For the others check the correlation between emitot and emicov and decide whether to better use this correlation for pccov\_fut.

\item[{‘corr’:}] \leavevmode
Take the correlation between emitot and emicov, unless it is too ‘bad’, then take the mean.

\end{description}

Default: ‘corr’.
\end{quote}

\sphinxstylestrong{all\_sectors\_covered(database, meta, ssp\_pccov, info\_per\_country, info\_for\_iso, iso3, ssp, cov\_gases, ssp\_kyoto\_ipcm0el)}
\begin{quote}

If all sectors are covered, one can calculate the pccov\_fut from the given share per gas
and the gases that are covered.
\end{quote}

\sphinxstylestrong{not\_all\_sectors\_covered(coverage, pccov\_his, preference\_pccov\_fut, info\_for\_iso,         iso3, ssp\_kyoto\_ipcm0el, ssp\_pccov, ssp, info\_per\_country, available\_years,         first\_year\_for\_slope, slope\_lim, rvalue\_lim)}
\begin{quote}

If not all sectors are covered check for the slope of the regression line to pccov\_his,
and if pccov\_his does not change too much use the mean (if preference\_pccov\_fut = ‘mean’),
else use the correlation between emitot and emicov.
\end{quote}

\sphinxstylestrong{data\_to\_database(ssp\_pccov, database, meta, ssp, primap)}
\begin{quote}

Put the data to ‘database’.
pccov, pcnotcov, emicov, and eminotcov.
\begin{quote}

If all sectors are covered, one can calculate the pccov\_fut from the given share per gas
and the gases that are covered.

If not all sectors are covered check for the slope of the regression line to pccov\_his,
and if pccov\_his does not change too much use the mean (if preference\_pccov\_fut = ‘mean’),
else use the correlation between emitot and emicov.
\end{quote}

Put the data to ‘database’.
pccov, pcnotcov, emicov, and eminotcov.
\end{quote}


\chapter{MODIFY\_INPUT\_HERE.input\_default}
\label{\detokenize{code:modify-input-here-input-default}}
Provide input for main\_ndc\_quantifications.py

UNITS:
Units for time series: emissions in Mt CO2eq, population in Pers, GDP in 2011GKD.
Units of NDC input data (infos\_from\_ndcs\_default.xlsx): Mt CO2eq or tCO2eq / cap (all AEI targets are in emissions per capita).

PREPROCESSING:
If you want to do the preprocessing of data, run preprocessing.py, and update the ‘folder\_preprocess’ in setup\_metadata.py.
Else, the folder stored in setup\_metadata.py will be used.

Chose the method for the pathway calculations (per country pathways).
\begin{description}
\item[{‘constant\_percentages’:}] \leavevmode
The percentage difference to the baseline emissions of the last available target year
is kept constant.

\item[{‘constant\_emissions’:}] \leavevmode
The emissions of the last available target year are kept constant.

\end{description}

Default:
meta.method\_pathways = ‘constant\_percentages’

For which countries should targets be used for the calculation of emission pathways for group of countries?
For the others, the baseline emissions will be used.

countries: ‘all’, or e.g., {[}‘EU28’, ‘AUS’, ‘CHN’{]}, or e.g., get\_isos\_groups({[}‘ANNEXI’{]}).

Default:
meta.calculate\_targets\_for = \{‘use\_it’: True, ‘countries’: ‘all’\}

Example:
cat\_countries = {[}‘ARG’, ‘AUS’, ‘BTN’, ‘BRA’, ‘CAN’, ‘CHL’, ‘CHN’, ‘CRI’, ‘EU28’,
\begin{quote}

‘ETH’, ‘GMB’, ‘IND’, ‘IDN’, ‘JPN’, ‘KAZ’, ‘KEN’, ‘MEX’, ‘MAR’,
‘NPL’, ‘NZL’,’NOR’, ‘PER’, ‘PHL’, ‘RUS’, ‘SAU’, ‘SGP’, ‘ZAF’,
‘KOR’, ‘CHE’, ‘TUR’, ‘ARE’, ‘USA’, ‘UKR’, ‘VNM’{]}
\end{quote}

meta.calculate\_targets\_for = \{‘use\_it’: True, ‘countries’: cat\_countries\}

Which target\sphinxhyphen{}types should be prioritised in the calculation of group\sphinxhyphen{}pathways?

ndcs\_type\_prioritisations can be a certain target tpye (e.g., ‘ABS’), or ‘TYPE\_ORIG’ or ‘TYPE\_CALC’.
Or several ordered options (only makes sense for != TYPE\_ORIG and != TYPE\_CALC).

One can chose from {[}‘TYPE\_ORIG’, ‘TYPE\_CALC’, ‘ABS’, ‘RBY’, ‘RBU’, ‘ABU’, ‘REI’, ‘AEI’{]}.

If TYPE\_ORIG: use the ‘original target type’ (what has been stated (+/\sphinxhyphen{})
in the NDC as target type).
If TYPE\_CALC: use the target type that has been assessed to be the ‘best
suitable’ (based on the NDC).
Explanation: e.g., when it is an RBU target, but the absolute target emissions are available
(e.g., given value, or based on their BAU and \%\sphinxhyphen{}reduction),
TYPE\_ORIG can be RBU, and TYPE\_CALC can be ABS. It can also have TYPE\_ORIG is
NGT and TYPE\_CALC is ABU, as they quantified some reductions.
Iterating through ndc\_type\_prioritisations, and using TYPE\_CALC if none of the iterations
found target values for the current target type in the NDC input file.
If ‘countries’ is ‘all’, apply it to all countries. Else, give ISO3s, and it
is only applied to those countries.
Else, the pathway is calculated based on TYPE\_CALC.

Default:
meta.ndcs\_type\_prioritisations = \{‘use\_it’: True, ‘ndcs\_type\_prioritisations’: {[}‘TYPE\_CALC’{]}, ‘countries’: ‘all’\}

Use NDC emissions data if available.
If TYPE\_CALC is used set it to True, for TYPE\_ORIG set it to False.

For countries without unconditional target: use the baseline emissions for the unconditional
pathway even if the conditional target is worse than the baseline (in 2030)?

Default:
meta.use\_baseline\_for\_uncondi\_even\_if\_baseline\_is\_better\_than\_condi = False

The targets are strengthened by ndc\_strengthen.

Chose between ‘how\_to’: ‘add’ or ‘multiply’.
‘add’: the reduction is increased by adding the value in ‘pc’.
‘multiply’: the reduction is increased by multiplying with (100\% + the value in ‘pc’).

If this results in a \% reduction that exceeds 100\%, it is set to 100\% (meaning a total
reduction of the covered part of emissions).

For absolute targets (ABS, ABU, AEI), it is not distinguished between ‘add’ and ‘multiply’.

Default:
meta.strengthen\_targets = \{‘use\_it’: False\}

Predefine that the coverage used for the pathways is 100\% for certain countries.
Only possible for relative targets / reductions.

Default:
meta.set\_pccov\_to\_100 = \{‘use\_it’: False\}

Groups for which to get the pathways.
‘EU28’, ‘EARTH’, ‘R5ASIA’, ‘R5LAM’, ‘R5MAF’, ‘R5OECD’, ‘R5REF’ will be calculated per default.
The R5 regions are needed for the temperature pathways (PRIMAP Emissions Module / Climate Module).

One can chose some of the following groups:
‘ANNEXI’, ‘ANNEXI\_KAZ’, ‘AOSIS’, ‘AR5’, ‘AG’, ‘BRICS’, ‘EIG’, ‘G7’, ‘G20’, ‘G77’,
‘GRADUATED\_LDCS’, ‘IMO’, ‘LDC’, ‘LLDC’, ‘NON\_ANNEXI’, ‘OECD’, ‘OPEC’, ‘SIDS’,
‘UMBRELLA’, ‘UNFCCC’, ‘UN\_REGIONAL\_GROUPS’, ‘AILAC’, ‘ALBA’, ‘APG’, ‘BASIC’, ‘CACAM’, ‘CD’,
‘CfRN’, ‘CVF’, ‘EEG’, ‘EIT’, ‘G77+China’, ‘GRULAC’, ‘KYOTO’, ‘LAS’, ‘LMDC’, ‘PA’,
‘SICA’, ‘UN’, ‘WEOG’
And provide a list for
meta.groups\_for\_which\_to\_calculate\_pathways

Default:
meta.groups\_for\_which\_to\_calculate\_pathways = {[}{]}


\chapter{\_to\_be\_run}
\label{\detokenize{code:to-be-run}}
Script to run the NDC quantifications (main\_ndc\_quantifications).
This includes per\sphinxhyphen{}country target emissions, pathways, and globally aggregated pathways.

Put in the name(s) of the wanted input\sphinxhyphen{}file(s) here. You can run several input\sphinxhyphen{}files in a row.

Default:
main\_ndc\_quantifications(‘input\_SSP2\_typeCalc\_pccov100’, ‘’)


\chapter{main\_ndc\_quantifications}
\label{\detokenize{code:main-ndc-quantifications}}
\sphinxstylestrong{main\_ndc\_quantifications(input\_file, lulucf\_prio)}
\begin{quote}

Calculation of NDC mitigation target emissions. Main file.

\# ————————\sphinxhyphen{}

Input examples:
\begin{quote}

input\_file = ‘input\_SSP2\_typeCalcForAllCountries’ (name of input\sphinxhyphen{}file, stored in /MODIFY\_INPUT\_HERE).

lulucf\_prio = ‘’ or ‘UNFCCC’ or ‘FAO’.
\begin{itemize}
\item {} 
If it is ‘’ the default LULUCF prioritisation is used (CRF, BUR, UNFCCC, FAO).

\item {} 
For ‘UNFCCC’: UNFCCC, CRF, BUR, FAO.

\item {} 
For ‘FAO’: FAO, CRF, BUR, UNFCCC.

\end{itemize}
\end{quote}

\# ————————\sphinxhyphen{}

Per country, target year, target type, conditionality, and range, the target emissions and emissions pathways are calculated.
The target is calculated once including, and excluding LULUCF.

\# ————————\sphinxhyphen{}

Target types
\begin{description}
\item[{ABS:}] \leavevmode
Absolute target emissions.
E.g., target is to reduce emissions in 2030 to 500 MtCO2eq.

\item[{RBY:}] \leavevmode
Relative reduction compared to base year.
E.g., 20\% emissions reduction compared to 2010 emissions in 2030.

\item[{RBU:}] \leavevmode
Relative reduction compared to BAU.
E.g., 20\% emissions reduction compared to business\sphinxhyphen{}as\sphinxhyphen{}usual (BAU) emissions in 2030.

\item[{ABU:}] \leavevmode
Absolute reduction compared to BAU.
E.g., 350 MtCO2eq reduction compared to BAU emissions in 2030.

\item[{REI:}] \leavevmode
Relative emissions intensity reduction.
Compared to base year. E.g., 20\% emissions intensity reduction compared to 2010 emissions intensity in 2030.
Or compared to BAU. This is basically a simple RRB target, but some NDCs state it as intensity targets.

\item[{AEI:}] \leavevmode
Absolute emissions intensity.
E.g., 2.1 tCO2eq/cap in 2030.

\item[{NGT:}] \leavevmode
Non\sphinxhyphen{}GHG targets.
Nothing is calculated, baseline emissions are assumed.

\end{description}

\# ————————\sphinxhyphen{}

Per country one target type is chosen for the aggregation to a global pathway.
This is generally the type\_orig or type\_calc.
\begin{description}
\item[{type\_orig: what is said (+/\sphinxhyphen{}) in the NDC, e.g., 20\% reduction compared to BAU (RBU).}] \leavevmode
In this case the comparison emissions are prioritised (comparison runs with ‘external’ input data).

\item[{type\_calc: what seems more suitable for the pathway calculations, e.g., if for the BAU target a quantification is given in the NDC (ABS).}] \leavevmode
In this case the emissions given in the NDCs are prioritised.

\end{description}

\# ————————\sphinxhyphen{}

Globally and regionally aggregated emissions pathways are given with the output suitable for the MATLAB PRIMAP Emissions / Climate Module.
They can be used to derive 2100 temperature estimates corresponding to the calculated 1990\sphinxhyphen{}2030 pathways.

Read input time series 1990\sphinxhyphen{}2050. For all countries in meta.isos.EARTH.
For the EU28 countries, the targets are calculated separately (using the EU28 NDC info for each of the countries).
In the end, a pathway is calculated for EU28.
\begin{description}
\item[{Read tables:}] \leavevmode
‘{\color{red}\bfseries{}KYOTOGHG\_IPCM0EL\_TOTAL\_NET\_}’ + meta.ssps.chosen + ‘FILLED\_PMSSPBIE’
‘KYOTOGHG\_IPCMLULUCF\_TOTAL\_NET\_INTEREXTRAPOL\_VARIOUS\{lulucf\_prio\}’
‘{\color{red}\bfseries{}KYOTOGHG\_IPCM0EL\_COV\_PC\_}’ + meta.ssps.chosen + ‘FILLED\_COVERAGE’
‘KYOTOGHG\_IPCMLULUCF\_COV\_EMI\_HISFUT\_COVERAGE’
‘{\color{red}\bfseries{}POP\_DEMOGR\_TOTAL\_NET\_}’ + meta.ssps.chosen + ‘FILLED\_PMSSPBIEMISC’
‘{\color{red}\bfseries{}GDPPPP\_ECO\_TOTAL\_NET\_}’ + meta.ssps.chosen + ‘FILLED\_PMSSPBIEMISC’

\end{description}

In principle one can also provide other tables (same type of data, but from other sources / with other values),
as long as they have the same structure and names.

If it is type\_calc: use the ndc\sphinxhyphen{}emissions if available.
For type\_orig: use the comparison data.
\begin{description}
\item[{Additionally calculate and add the following tables to the database:}] \leavevmode
‘{\color{red}\bfseries{}KYOTOGHG\_IPCM0EL\_NOTCOV\_PC\_}’ + meta.ssps.chosen + ‘FILLED\_COVERAGE’
‘{\color{red}\bfseries{}KYOTOGHG\_IPCM0EL\_COV\_EMI\_}’ + meta.ssps.chosen + ‘FILLED\_COVERAGE’
‘{\color{red}\bfseries{}KYOTOGHG\_IPCM0EL\_NOTCOV\_EMI\_}’ + meta.ssps.chosen + ‘FILLED\_COVERAGE’
‘{\color{red}\bfseries{}KYOTOGHG\_IPC0\_TOTAL\_NET\_}’ + meta.ssps.chosen + ‘\_VARIOUS\{lulucf\_prio\}’

\end{description}

Write relevant input\sphinxhyphen{}info to log\_file.md in the output\sphinxhyphen{}folder.
In this file one will find the necessary information to re\sphinxhyphen{}run the calculations with the same setup.

Correct or modify the calculation options given in input\_file if necessary.
ndcs\_check\_options\_for\_target\_calculations(meta).

TARGETS
Calculate the target emissions per country and un/conditional \& best/worst \& year.
ndcs\_calculate\_targets(database, meta).

PATHWAYS
Country\sphinxhyphen{}pathways.

Calculate the emissions pathways for un/conditional\_best/worst targets, per country.
For countries without targets use the baseline emissions (if available).
ndcs\_calculate\_pathways\_per\_country(database, calculated\_targets, meta)

PATHWAYS
Group\sphinxhyphen{}pathways.

Calculate the emissions pathways for un/conditional\_best/worst targets, per group of countries.
ndcs\_calculate\_pathways\_per\_group(pathways\_per\_country, meta)
\end{quote}


\chapter{main\_functions.ndcs\_some\_initial\_testing}
\label{\detokenize{code:main-functions-ndcs-some-initial-testing}}
\sphinxstylestrong{ndcs\_some\_initial\_testing(database, meta)}
\begin{quote}

Check input data for ‘how many countries have no / missing data’ and ‘are the values of pc\_cov/ncov between 0 and 1’.
Does not check whether the emissions / pop / gdp data seem realistic.
\end{quote}


\chapter{main\_functions.ndcs\_check\_options\_for\_target\_calculations}
\label{\detokenize{code:main-functions-ndcs-check-options-for-target-calculations}}
\sphinxstylestrong{ndcs\_check\_options\_for\_target\_calculations(meta)}
\begin{quote}

Check whether the values for attributes of classes meta.calculate\_targets\_for, meta.ndcs\_type\_prioritisations,
meta.set\_pccov\_to\_100, and meta.strengthen\_targets \textendash{} given in input\_file \textendash{} are ok.
\end{quote}

\sphinxstylestrong{check\_countries(countries, meta)}
\begin{quote}

Check for valid country codes.
\end{quote}


\chapter{main\_functions.ndcs\_calculate\_targets}
\label{\detokenize{code:main-functions-ndcs-calculate-targets}}
\sphinxstylestrong{ndcs\_calculate\_targets(database, meta)}
\begin{quote}

Calculate the NDC targets for each of the target types available for a country.
Some NDCs give more than one target type (e.g., they give the absolute value,
but say that it is a RBU target and therefore give the \% reduction against BAU).

EU28 countries are calculated separately, using the NDC information from the EU28 NDC.
The per\sphinxhyphen{}country target emissions for EU28 countries does not equal the ‘real’ emissions targets
(each of the countries has its own targets to in\sphinxhyphen{}sum get to the EU28 total target).
\end{quote}

\sphinxstylestrong{check\_ndc\_values(ict, iso\_act)}
\begin{quote}

Check if ndc\_values seem ok.
For the different target types there are some criteria to check if the ndc\_value seems plausible.
Does not cover every eventuality.
\end{quote}

\sphinxstylestrong{strengthen\_targets(meta, ict, ndc\_value\_exclLU, ndc\_value\_inclLU, iso\_act)}
\begin{quote}

Apply a strengthening to the targets.
Depends on ‘how\_to’ (‘multiply’ or ‘add’).

‘add’: the reduction is increased by adding the value in ‘pc’.
‘multiply’: the reduction is increased by multiplying with (100\% + the value in ‘pc’).
If this results in a \% reduction that exceeds 100\%, it is set to 100\% (meaning a total
reduction …).

For absolute targets (ABS, ABU, AEI), it is not distinguished between ‘add’ and ‘multiply’.
\end{quote}

\sphinxstylestrong{quantification\_per\_country(iso\_act, meta, lulucf\_first\_try, calculated\_targets, txt)}
\begin{quote}

Per country get the information on the different available target types, and get the calculation data for
emi, pop, gdp for the reference and target years.
Calculate the single target emissions, stored in ndc\_targets.csv.
\begin{quote}

Get all available target years \& un/conditional \& best/worst targets.
ndc\_value depends on the target type, either it is an absolute emission (ABS, ABU),
a percentage for relative reductions (RBY, RBU, REI), or an emissions intensity (AEI).
\begin{itemize}
\item {} 
ABS: target emissions in MtCO2eq.

\item {} 
ABU: absolute  reduction in MtCO2eq (e.g., \sphinxhyphen{}3500 stands for 3500 MtCO2eq reduction).

\item {} 
RBY, RBU, REI: percentage reduction (e.g., \sphinxhyphen{}20 stands for 20\% reduction).

\item {} 
AEI: emissions intensity in target year, e.g., 3.2 stands for 3.2 t/cap or 3.2 t/GDP, when int\_ref is POP or GDP, respectively.

\end{itemize}

In infos\_from\_ndcs\_default.xlsx, infos\_from\_ndcs.csv, and meta.ndcs\_info,
the ‘available targets’ per target type are stored as nested dictionaries and can be read in using json.
Targets are also separated in ‘inclLU’ and ‘exclLU’.
\begin{quote}

Iterate through the combinations of target years \& un/conditional \& best/worst.
Do the calculations depending on the ndc values and the target type.
\begin{quote}

Set the coverage to 100\% (not changing LULUCF), if meta.set\_pccov\_to\_100 is True and the country is in the ‘wanted’ countries.
Only for relative targets / reductions.

Check if ndc\_values seem ok.
For the different target types there are some criteria to check if the ndc\_value seems plausible.
Does not cover every eventuality.

Get the numerical values from ndc\_value\_exclLU and ndc\_value\_inclLU.
exclLU or inclLU was assessed based on the NDCs and stored together with the targets
(reductions or absolute values).
exclLU was chosen if LULUCF is not covered, or if LULUCF was covered, but it was stated that this value is for exclLU.
inclLU was chosen if LULUCF is covered, or if LULUCF was not covered and it was stated that this value is for inclLU.

Convert the given absolute value to AR4 (no conversion factors for AR5 available).
Based on national conversion factors from AR2 to AR4 from PRIMAPHIST21 KYOTOGHG\_IPCM0EL.

If strengthen NDC is chosen apply the strengthening to the reductions.

Calculate the mitigated target year emissions depending on the target type.
\end{quote}
\end{quote}
\end{quote}
\end{quote}

\sphinxstylestrong{calculate\_targets\_depending\_on\_type(        ict, iso\_act, refyr, taryr, ndc\_value\_exclLU, ndc\_value\_inclLU, meta, lulucf\_first\_try, txt)}
\begin{quote}

Calculation of targets: excluding LULUCF and including LULUCF.
For one country, target type, target year, conditionality, range.

Get the emissions values from NDCs.

Calculate the NDC\sphinxhyphen{}level for relative targets. E.g., target is \sphinxhyphen{}20\%, so the level is 80\% (100\%\sphinxhyphen{}20\%).

Get the information on whether LULUCF is included in the target or not.

Get the LULUCF data for the reference and target year.
\end{quote}

\sphinxstylestrong{calc\_targets\_inclLU\_exclLU(tar\_emi\_exclLU, tar\_emi\_inclLU, bl\_onlyLU\_taryr, tar\_type\_used, txt)}
\begin{quote}
\begin{quote}
\begin{quote}

If not both, inclLU and exclLU information are given:
calculate the ‘other case’ from the given case.

If no inclLU target is given, calculate it as the sum over tar\_emi\_exclLU and bl\_LU.

Options to calculate the exclLU target from the given inclLU target:
\begin{quote}
\begin{description}
\item[{meta.prio\_tar\_exclLU\_from\_tar\_inclLU == ‘apply\_rel\_red\_to\_emi\_exclLU’:}] \leavevmode
If no exclLU target is given, calculate it assuming the same \% reductions in all sectors.
Calculate the \% reduction of ABS\_inclLU compared to the bl\_inclLU (use the value given in NDC, if possible)
and apply it to our emi\_bl\_exclLU (assuming 100\% coverage).

\item[{meta.prio\_tar\_exclLU\_from\_tar\_inclLU == ‘subtract\_LU\_in\_taryr’:}] \leavevmode
If no exclLU target is given, subtract the target year onlyLU estimate from the inclLU target (like CAT).

\end{description}

If it needs a second try due to the LULUCF part (if the exclLU target is negative, when calculate from the inclLU target):
\begin{quote}

Get the ABU\_inclLU and split it into the onlyLU and exclLU parts
(depending on the onlyLU and exclLU contributions in the target year).

tar\_exclLU: bl\_exclLU + ABU\_exclLU (ABU\_exclLU is negative).
\end{quote}
\end{quote}
\end{quote}

ABS target: tar\_emi is the given absolute emissions value.

ABU target: tar\_emi is the given baseline minus the absolute reduction.

AEI target: tar\_emi is the given absolute emissions intensity multiplied by the target year GDP or POP.

RBY, RBU or REI target.
\begin{quote}

REI compared to BAU is same as RBU.
For REI\_RBY, the emissions intensities must be used.

inclLU, LU is covered: include the emi\_bl\_onlyLU\_refyr emissions in the target.
If emi\_bl\_onlyLU\_refyr is negative, only apply the reduction to the exclLU\sphinxhyphen{}part.
And add the bl\_onlyLU\_refyr as is.
\end{quote}
\end{quote}

Iterate through the iso3s in meta.ndcs\_info (EU28 as single countries, using the EU28 target info).
Calculate ‘all available targets’ per country (targets as in meta.ndcs\_info, in columns
{[}‘ABS’, ‘RBY’, ‘RBU’, ‘ABU’, ‘REI’, ‘AEI’{]}).
\begin{quote}

First run of quantifications.

If any of the tar\_exclLU are negative re\sphinxhyphen{}run the quantifications with different method.
\end{quote}
\end{quote}


\chapter{main\_functions.ndcs\_calculate\_pathways\_per\_country}
\label{\detokenize{code:main-functions-ndcs-calculate-pathways-per-country}}
\sphinxstylestrong{ndcs\_calculate\_pathways\_per\_country(database, calculated\_targets, meta)}
\begin{quote}

Calculate the pathway per country and target (per un/conditional best/worst).
Only for IPCM0EL.

Stored in ‘ndc\_targets\_pathways\_per\_country.csv’.
\end{quote}

\sphinxstylestrong{add\_timeseries(meta, database, ptws\_all)}
\begin{quote}

Get the cydata for all countries (are stored in ts\_act.data) and add some information (gwp, category, …).
\end{quote}

\sphinxstylestrong{add\_targets\_from\_other\_conditionality\_if\_needed(meta, targets\_act)}
\begin{quote}

If there is a value in unconditional\_best in a year, but not in unconditional\_worst, put it there.
If there is a value in conditional\_best in a year, but not in conditional\_worst, put it there.
If there is a value in unconditional in a year, but not in conditional, put it there (the one
stored in ‘unconditional\_worst’).
\end{quote}

\sphinxstylestrong{calculate\_pathways(ptws\_all, emi\_bl\_act, targets\_act, current\_tar, curr\_tar, iso\_act, meta)}
\begin{quote}

Calculate the pathways per un/conditional\_best/worst, assuming a linear in/decrease of the
\% difference to the baseline emissions.
E.g., if the country has one unconditional\_best target that equals a reduction of 20\% in 2030
(compared to baseline emissions):
the pathway has a 0\% reduction in meta.years.pathways{[}0{]}, and in 2030 the given
20\% reduction, with a linear increase between meta.years.pathways{[}0{]} and 2030.

Use the baseline emissions for 2020.
Calculate the percentage level (how much percent of the baseline emissions do
the single target emissions represent).
pc\_level: interpolate between available values, and keep the last available value of pc\_level constant
(if meta.method\_pathways == ‘constant\_percentages’), or keep the last emissions level constant
(if meta.method\_pathways == ‘constant\_emissions’).

E.g., 2020 is 100\%, 2030 is 80\%. For 2025 it is 100\% + (80\%\sphinxhyphen{}100\%)/(2030\sphinxhyphen{}2020) * (2025\sphinxhyphen{}2020) = 90\%.
All values of pc\_level are 100\% (= baseline emissions) if there are no available\_years.
baseline emissions are used in years before the first meta.years.pathways.
Apply pc\_level to baseline emissions to calculate the pathway.
\begin{quote}

Keep pc\_level constant after last available target for targets that are below the baseline of that year.
\end{quote}

If the last target is higher than the baseline, use the same growth as in the baseline.
Else, the mitigated pathway can increase a lot.

Keep the emissions constant after last available target if meta.method\_pathways == ‘constant\_emissions’.
Else the pc\_level is kept constant.

If the country has an entry for ‘DECLINE\_AFTER\_YEAR’: stop the mitigated emissions from increasing after that year.
\end{quote}

\sphinxstylestrong{add\_baseline\_emissions(emi\_bl\_act, meta, iso\_act, columns)}
\begin{quote}
\begin{quote}

Add the baseline emissions for each iso3 (as un/conditional best/worst pathways).
\end{quote}

Get the cydata for all countries (are stored in ts\_act.data) and add some information (gwp, category, …).

Iterate through EARTH countries, EU28 as single countries.
Calculate the pathway per country and target type and un/conditional best/worst.
Use baseline emissions where necessary.
Use the unconditional target as conditional target as well, if the country does not have a conditional target.

If meta.use\_baseline\_for\_uncondi\_even\_if\_baseline\_is\_better\_than\_condi = True
use the baseline emissions for the unconditional pathways even if the conditional
pathway (for 2030) is worse than the baseline.
Else use the conditional pathway as unconditional pathway as well.
\begin{quote}
\begin{quote}

If nothing is available, use ‘baseline\_emissions’.

Calculate pathways per target type.
\begin{quote}

If there is a value in unconditional\_best in a year, but not in unconditional\_worst, put it there.
If there is a value in conditional\_best in a year, but not in conditional\_worst, put it there.
If there is a value in unconditional in a year, but not in conditional, put it there (the one
stored in ‘unconditional\_worst’).

Calculate the pathways per un/conditional\_best/worst, assuming a linear in/decrease of the
\% difference to the baseline emissions.
E.g., if the country has one unconditional\_best target that equals a reduction of 20\% in 2030
(compared to baseline emissions):
the pathway has a 0\% reduction in meta.years.pathways{[}0{]}, and in 2030 the given
20\% reduction, with a linear increase between meta.years.pathways{[}0{]} and 2030.

Check for meta.use\_baseline\_for\_uncondi\_even\_if\_baseline\_is\_better\_than\_condi.
If it is False and the conditional\_worst pathway is worse in 2030 than the baseline
put the conditional\_worst pathway to the unconditional pathways.
\end{quote}
\end{quote}

For each country, even if it does have a different target, also give out ‘baseline\_emissions’.
\end{quote}
\end{quote}


\chapter{main\_functions.ndcs\_calculate\_pathways\_per\_group}
\label{\detokenize{code:main-functions-ndcs-calculate-pathways-per-group}}
\sphinxstylestrong{ndcs\_calculate\_pathways\_per\_group(    pathways\_per\_country, meta)}
\begin{quote}

Calculate the emissions pathways for a group of countries by summing up all available per\sphinxhyphen{}country pathways,
per un/conditional\_best/worst (chosing one of the target types per country).
And using baseline emissions where no target is available.

Chosen per\sphinxhyphen{}country pathways stored in ndc\_targets\_pathways\_per\_country\_used\_for\_group\_pathways.csv.
Group pathways stored in ndc\_targets\_pathways\_per\_group.csv.
\end{quote}

\sphinxstylestrong{get\_pathways\_per\_country\_used\_for\_group\_pathways(        input\_pathways, pathways\_per\_country, meta)}
\begin{quote}

Get the ‘chosen’ country\sphinxhyphen{}pathways that will be used for the group pathways.
Depend on the preferred (and available) target types.

Chosen per\sphinxhyphen{}country pathways stored in ndc\_targets\_pathways\_per\_country\_used\_for\_group\_pathways.csv.
\end{quote}

\sphinxstylestrong{calc\_pathways\_of\_group(input\_pathways, ptws\_groups, ptws\_chosen\_per\_country,         meta, group\_act, tars\_to\_use)}
\begin{quote}

Calculate the group pathways (for un/conditional best/worst; using the chosen country\sphinxhyphen{}pathways).
\end{quote}


\chapter{main\_functions.ndcs\_some\_final\_testing}
\label{\detokenize{code:main-functions-ndcs-some-final-testing}}
\sphinxstylestrong{ndcs\_some\_final\_testing(database, meta)}
\begin{quote}

Doing some testing of the constructed global pathways.
\end{quote}

\sphinxstylestrong{check\_one()}
\begin{quote}

Check if the EARTH pathways for emi\_bau, pop, gdp seem ok.
\end{quote}

\sphinxstylestrong{check\_two()}
\begin{quote}

Check if the EARTH pathways for pc\_cov, emi\_cov seem ok.
\end{quote}

\sphinxstylestrong{check\_three()}
\begin{quote}

Check if the ‘used’ EARTH pathways seem ok.
\end{quote}

\sphinxstylestrong{check\_four()}
\begin{quote}

Check if the target pathways for EARTH seem ok.
Countries for which something (some emissions based on data provided within the NDC  or target emissions) were calculated: ndcs\_\_target\_calculations\_for\_input\_xlsx.
\end{quote}



\renewcommand{\indexname}{Index}
\printindex
\end{document}